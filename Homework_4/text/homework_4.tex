\documentclass[a4paper, 12pt]{article}

%Ukrainian language

\usepackage[T1,T2A]{fontenc}
\usepackage[utf8]{inputenc}
\usepackage[english,ukrainian]{babel}

%Math
\usepackage{amsmath,amsfonts,amssymb,amsthm,mathtools} 

% Images
\usepackage{graphicx}
\usepackage{wrapfig}

%Plots
\usepackage{tikz}
\usetikzlibrary{positioning}
\usetikzlibrary{patterns}

%Title
\author{Демедюк Віталій}
\title{Дослідження операцій\\
	   Домашня робота №4}
\date{\today}

%Text color
\usepackage{xcolor}

%Multirow
\usepackage{multirow}

\usepackage{float}

\usepackage[pdftex,
colorlinks,%
linkcolor=blue,citecolor=red,urlcolor=blue,
hyperindex,%
plainpages=false,%
bookmarksopen,%
bookmarksnumbered,%
unicode]{hyperref}

\begin{document}

\maketitle
\newpage

\tableofcontents


\newpage

\section{Пряма задача $\rightarrow$ двоїста задача}

\subsection{Пряма задача}

Цільова функція:

\begin{equation*}
L = 2x_1 + x_2 \rightarrow \max
\end{equation*}
Обмеження:

\begin{equation*}
\left\{
\begin{aligned}
	x_1 &+ 2x_2 \leq 7,\\
	5x_1 &+ 7x_2 \geq 4,\\
	3x_1 &- 2x_2 \leq 10,\\
	x_1 &\geq 0, x_2 \geq 0. 
\end{aligned}
\right.
\Longleftrightarrow
\left\{
\begin{aligned}
	x_1 &+ 2x_2 \leq 7,\\
	-5x_1 &- 7x_2 \leq -4,\\
	3x_1 &- 2x_2 \leq 10,\\
	x_1 &\geq 0, x_2 \geq 0. 
\end{aligned}
\right.
\end{equation*}

\subsection{Двоїста задача}

Цільова функція:

\begin{equation*}
F = 7y_1 - 4y_2 + 10y_3 \rightarrow \min
\end{equation*}
Обмеження:

\begin{equation*}
\left\{
\begin{aligned}
	y_1 &- 5y_2 + 3y_3 \geq 2,\\
	2y_1 &- 7y_2 - 2y_3 \geq 1,\\
	y_1 &\geq 0, y_2 \geq 0, y_3 \geq 0. 
\end{aligned}
\right.
\end{equation*}

\section{Двоїста ЗЗЛП $\rightarrow$ двоїста КЗЛП}

\subsection{Двоїста ЗЗЛП $\rightarrow$ двоїста СЗЛП}


Цільова функція:

\begin{equation*}
F = 7y_1 - 4y_2 + 10y_3 + 0y_4 + 0y_5 \rightarrow \min
\end{equation*}

Обмеження:

\begin{equation*}
\left\{
\begin{aligned}
	y_1 &- 5y_2 + 3y_3 - y_4 \textcolor{white}{+ 0y_5} = 2,\\
	2y_1 &- 7y_2 - 2y_3 \textcolor{white}{+ 0y_4} - y_5 = 1.
\end{aligned}
\right.
\end{equation*}

\[ y_1, y_2, y_3, y_4, y_5 \geq 0 \]


\subsection{Двоїста СЗЛП $\rightarrow$ двоїста КЗЛП (М-задача)}

Запишемо СЗЛП у векторній формі

\begin{equation*}
\overline{y} = \left(y_1, y_2, y_3, y_4, y_5\right)^T \text{ -- вектор-стовпець змінних}
\end{equation*}

\begin{equation*}
\overline{b} = \left(b_1, b_2, b_3, b_4, b_5\right) = (7,-4,10,0,0) \text{ -- вектор коефіціентів у фунції } F
\end{equation*}

\begin{equation*}
F = \overline{b}\overline{y} \rightarrow \min
\end{equation*}

$A^T - \text{матриця коефіціентів системи обмежень}$

\begin{equation*}
 A^T = \Vert a_{ji} \Vert = 
\begin{pmatrix}
	1 & -5 & 3 & -1 & 0\\
	2 & -7 & -2 & 0 & -1
\end{pmatrix}
\end{equation*}

$ \overline{c} - \text{вектор, що } A^T\overline{y} = \overline{c} $

\begin{equation*}
 \overline{c} = \left(2, 1 \right)^T
\end{equation*}

В КЗЛП повинні виконуватися наступні умови:

$\overline{c} \geqslant \overline{0} \text{, } \overline{y} \geqslant \overline{0} \text{, } A^T\overline{y} = \overline{c} \text{, } A^T - \text{містить одиничну підматрицю}$\\

Можемо побачити, що у нас не виконується остання умова, тому скористаємся М-методом, щоб добавити штучний базис та отримати М-задачу з початковим базисним роз'язком\\

\begin{equation*}
F' = 7y_1 - 4y_2 + 10y_3 + 0y_4 + 0y_5 + M(w_1 + w_2) \rightarrow \min
\end{equation*}


\begin{equation*}
\left\{
\begin{aligned}
	y_1 &- 5y_2 + 3y_3 - y_4 \textcolor{white}{+ 0y_5} + w_1 \textcolor{white}{+ 0w_2} = 2,\\
	2y_1 &- 7y_2 - 2y_3 \textcolor{white}{+ 0y_4} - y_5 \textcolor{white}{+ 0w_1} + w_2 = 1.
\end{aligned}
\right.
\end{equation*}

Векторна форма

\begin{equation*}
\overline{y'} = \left(y_1, y_2, y_3, y_4, y_5, w_1, w_2\right)^T
\end{equation*}

\begin{equation*}
\overline{b'} = \left(b_1, b_2, b_3, b_4, b_5, b_6, b_7\right) = (7,-4,10,0,0,M,M)
\end{equation*}

\begin{equation*}
F' = \overline{b'}\overline{y'} \rightarrow \min
\end{equation*}

\begin{equation*}
A^T{'} = \Vert a_{ji} \Vert = 
\begin{pmatrix}
	1 & -5 & 3 & -1 & 0 & 1 & 0\\
	2 & -7 & -2 & 0 & -1 & 0 & 1
\end{pmatrix}
\end{equation*}

\begin{equation*}
 \overline{c} = \left( 2, 1 \right)^T
\end{equation*}

\begin{equation*}
 A^T{'}\overline{y'} = \overline{c}
\end{equation*}


\section{Симплекс-метод}

\subsection{Розв'язок М-задачі симплекс-методом}

\subsubsection{Крок №1}

\begin{table}[H]
    \centering
    \begin{tabular}{|l|l|l|l|l|l|l|l|l|l|l|}
    \hline
         &  & $b_1$ & $b_2$ & $b_3$ & $b_4$ & $b_5$ & $b_6$ & $b_7$ & & \\ \hline
         &  & 7 & -4 & 10 & 0 & 0 & M & M & &  \\ \hline
         $b_\text{б}$ & $y'_\text{б}$ & $A^T{'}_1$ & $A^T{'}_2$ & $A^T{'}_3$ & $A^T{'}_4$ & $A^T{'}_5$ & $A^T{'}_6$ & $A^T{'}_7$ & $c$  & $\Theta$ \\ \hline
        M & $w_1$ & 1 & -5 & 3 & -1 & 0 & 1 & 0 & 2 & 2 \\ \hline
        M & $w_2$ & 2 & -7 & -2 & 0 & -1 & 0 & 1 & 1 & 1/2 \\ \hline
         & $\Delta_i$ & 7-3M & -4+12M & 10-M & M & M & 0 & 0 & &  \\ \hline
    \end{tabular}
\end{table}

\begin{equation*}
	\Delta_i = b_i - \sum_{j=1}^{2} b_\text{бj} \alpha_{ji}  
\end{equation*}

\begin{equation*}
	\Delta_k = \min_{i=1 \dots 7} \Delta_i = \Delta_1 = 7-3M
\end{equation*}

\begin{equation*}
	\Theta_t = \min_{j: \Theta_j \geqslant 0} \Theta_j = \Theta_2 = \frac{4}{7}
\end{equation*}

\begin{equation*}
	\Theta_t = \frac{b_t}{A^T{'}_{kt}} = \frac{b_2}{A^T{'}_{k2}}
\end{equation*}

$t = 2$ - ведучий рядок

$k = 1$ - ведучий стовпець

$a_{tk} = a_{21}$ - ведучий елемент

$l = 7$

$l$-у змінну виводимо з базису і вводимо $k$-у.

\subsubsection{Крок №2}

\begin{table}[H]
    \centering
    \begin{tabular}{|l|l|l|l|l|l|l|l|l|l|l|}
    \hline
         &  & $b_1$ & $b_2$ & $b_3$ & $b_4$ & $b_5$ & $b_6$ & $b_7$ & & \\ \hline
         &  & 7 & -4 & 10 & 0 & 0 & M & M & &  \\ \hline
         $b_\text{б}$ & $y'_\text{б}$ & $A^T{'}_1$ & $A^T{'}_2$ & $A^T{'}_3$ & $A^T{'}_4$ & $A^T{'}_5$ & $A^T{'}_6$ & $A^T{'}_7$ & $c$  & $\Theta$ \\ \hline
        M & $w_1$ & 0 & -3/2 & 4 & -1 & 1/2 & 1 & -1/2 & 3/2 & 3/8 \\ \hline
        7 & $y_1$ & 1 & -7/2 & -1 & 0 & -1/2 & 0 & 1/2 & 1/2 & \\ \hline
         & $\Delta_i$ & 0 & 41/2 + 3M/2 & 17-4M	& M & 7/2 - M/2 & 0 & 3M/2 - 7/2 & & \\ \hline
    \end{tabular}
\end{table}

\begin{equation*}
	\Delta_i = b_i - \sum_{j=1}^{2} b_\text{бj} \alpha_{ji}  
\end{equation*}

\begin{equation*}
	\Delta_k = \min_{i=1 \dots 7} \Delta_i = \Delta_3 = 17-4M
\end{equation*}

\begin{equation*}
	\Theta_t = \min_{j: \Theta_j \geqslant 0} \Theta_j = \Theta_2 = \frac{4}{7}
\end{equation*}

\begin{equation*}
	\Theta_t = \frac{b_t}{A^T{'}_{kt}} = \frac{b_2}{A^T{'}_{k1}}
\end{equation*}

$t = 1$ - ведучий рядок

$k = 3$ - ведучий стовпець

$a_{tk} = a_{13}$ - ведучий елемент

$l = 6$

$l$-у змінну виводимо з базису і вводимо $k$-у.

\subsubsection{Крок №3}

\begin{table}[H]
    \centering
    \begin{tabular}{|l|l|l|l|l|l|l|l|l|l|l|}
    \hline
         &  & $b_1$ & $b_2$ & $b_3$ & $b_4$ & $b_5$ & $b_6$ & $b_7$ & & \\ \hline
         &  & 7 & -4 & 10 & 0 & 0 & M & M & &  \\ \hline
         $b_\text{б}$ & $y'_\text{б}$ & $A^T{'}_1$ & $A^T{'}_2$ & $A^T{'}_3$ & $A^T{'}_4$ & $A^T{'}_5$ & $A^T{'}_6$ & $A^T{'}_7$ & $c$  & $\Theta$ \\ \hline
        10 & $y_3$ & 0 & -3/8 & 1 & -1/4 & 1/8 & 1/4 & -1/8 & 3/8 & \\ \hline
        7 & $y_1$ & 1 & -31/8 & 0 & -1/4 & -3/8 & 1/4 & 3/8 & 7/8 & \\ \hline
         & $\Delta_i$ & 0 & 303/8 & 0 & 45/4 & 67/8 & 39/4 & 45/8 & & \\ \hline
    \end{tabular}
\end{table}

Оскільки $\min_{i=1 \dots 6} \Delta_i \geqslant 0$, ми можемо завершити симплекс-метод

\subsubsection{Результат застосування симплекс-метода до двоїстоої задачі}


\begin{equation*}
\text{При }\overline{y'} = \left(y_1, y_2, y_3, y_4, y_5, w_1, w2 \right) = \left(\frac{7}{8}, 0, \frac{3}{8}, 0, 0, 0, 0\right)
\end{equation*}

\begin{equation*}
\text{функція }F' = 7y_1 - 4y_2 + 10y_3 + 0y_4 + 0y_5 + M(w_1 + w_2) \rightarrow \min,
\end{equation*}


\begin{equation*}
\text{отже }F = 7y_1 - 4y_2 + 10y_3 \rightarrow \min \text{, при } y_1 = \frac{7}{8}, y_2 = 0, y_1 = \frac{3}{8}
\end{equation*}

\subsection{ Перехід від розв'язку двоїстої ЗЛП до розв'язку прямої ЗЛП }

\begin{equation*}
	B = \left(A^T{'}_3, A^T{'}_1\right) = 
	\begin{pmatrix*}
		3 & 1\\
		-2 & 2
	\end{pmatrix*}
\end{equation*}

\begin{equation*}
	B^{-1} = 
	\begin{pmatrix*}
		\frac{1}{4} & -\frac{1}{8}\\
		\frac{1}{4} & \frac{3}{8}
	\end{pmatrix*}
\end{equation*}

\begin{equation*}
	x^* = (x_1, x_2) \text{ -- розв'язок прямої ЗЛП}.
\end{equation*}

\begin{equation*}
	x^* = b_{\text{б}}B^{-1} = (10,7)
	\begin{pmatrix*}
		\frac{1}{4} & -\frac{1}{8}\\
		\frac{1}{4} & \frac{3}{8}
	\end{pmatrix*} = \left(\frac{17}{4}, \frac{11}{8}\right)
\end{equation*}

\end{document}