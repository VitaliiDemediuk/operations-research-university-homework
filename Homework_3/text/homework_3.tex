\documentclass[a4paper, 12pt]{article}

%Ukrainian language

\usepackage[T1,T2A]{fontenc}
\usepackage[utf8]{inputenc}
\usepackage[english,ukrainian]{babel}

%Math
\usepackage{amsmath,amsfonts,amssymb,amsthm,mathtools} 

% Images
\usepackage{graphicx}
\usepackage{wrapfig}

%Plots
\usepackage{tikz}
\usetikzlibrary{positioning}
\usetikzlibrary{patterns}

%Title
\author{Демедюк Віталій}
\title{Дослідження операцій\\
	   Домашня робота №3}
\date{\today}

%Text color
\usepackage{xcolor}

%Multirow
\usepackage{multirow}

\usepackage{float}

\usepackage[pdftex,
colorlinks,%
linkcolor=blue,citecolor=red,urlcolor=blue,
hyperindex,%
plainpages=false,%
bookmarksopen,%
bookmarksnumbered,%
unicode]{hyperref}

\begin{document}

\maketitle
\newpage

\tableofcontents


\newpage

\section{ЗЗЛП $\rightarrow$ КЗЛП}

\subsection{ЗЗЛП}

Цільова функція:

\begin{equation*}
L = 2x_1 + x_2 \rightarrow \max
\end{equation*}
Обмеження:

\begin{equation*}
\left\{
\begin{aligned}
	x_1 &+ 2x_2 \leq 7,\\
	5x_1 &+ 7x_2 \geq 4,\\
	3x_1 &- 2x_2 \leq 10,\\
	x_1 &\geq 0, x_2 \geq 0. 
\end{aligned}
\right.
\end{equation*}

\subsection{ЗЗЛП $\rightarrow$ СЗЛП}

Цільова функція:

\begin{equation*}
L = -2x_1 - x_2 + 0x_3 + 0x_4 + 0x_5 \rightarrow \min
\end{equation*}
Обмеження:

\[
\left\{
\begin{aligned}
	x_1 &+ 2x_2 + x_3 \textcolor{white}{+ 0x_4 + 0x_5} = 7,\\
	5x_1 &+ 7x_2 \textcolor{white}{+ 0x_33} - x_4 \textcolor{white}{+ 0x_5} = 4,\\
	3x_1 &- 2x_2 \textcolor{white}{+ 0x_3 + 0x_4} + x_5 = 10.\\ 
\end{aligned}
\right.
\]

\[ x_1, x_2, x_3, x_4, x_5 \geq 0 \]


\subsection{СЗЛП $\rightarrow$ КЗЛП (М-задача)}

Запишемо СЗЛП у векторній формі

\begin{equation*}
\overline{x} = \left(x_1, x_2, x_3, x_4, x_5\right)^T \text{ -- вектор-стовпець змінних}
\end{equation*}

\begin{equation*}
\overline{c} = \left(c_1, c_2, c_3, c_4, c_5\right) = (-2,-1,0,0,0) \text{ -- вектор коефіціентів у фунції } L
\end{equation*}

\begin{equation*}
L = \overline{c}\overline{x} \rightarrow \min
\end{equation*}

$A - \text{матриця коефіціентів системи обмежень}$

\begin{equation*}
 A = \Vert a_{ij} \Vert = 
\begin{pmatrix}
	1 & 2 & 1 & 0 & 0\\
	5 & 7 & 0 & -1 & 0\\
	3 & -2 & 0 & 0 & 1
\end{pmatrix}
\end{equation*}

$ \overline{b} - \text{вектор, що } A\overline{x} = \overline{b} $

\begin{equation*}
 \overline{b} = \left( 7, 4, 10 \right)^T
\end{equation*}

В КЗЛП повинні виконуватися наступні умови:

$\overline{b} \geqslant \overline{0} \text{, } \overline{x} \geqslant \overline{0} \text{, } A\overline{x} = \overline{b} \text{, } A - \text{містить одиничну підматрицю}$\\

Можемо побачити, що у нас не виконується остання умова, тому скористаємся М-методом, щоб добавити штучний базис та отримати М-задачу з початковим базисним роз'язком\\

\begin{equation*}
L' = -2x_1 - x_2 + 0x_3 + 0x_4 + 0x_5 + M(y_1) \rightarrow \min
\end{equation*}


\[
\left\{
\begin{aligned}
	x_1 &+ 2x_2 + x_3 \textcolor{white}{+ 0x_4 + 0x_5 + y_1} = 7,\\
	5x_1 &+ 7x_2 \textcolor{white}{+ 0x_33} - x_4 \textcolor{white}{+ 0x_5} + y_1 = 4,\\
	3x_1 &- 2x_2 \textcolor{white}{+ 0x_3 + 0x_4} + x_5 \textcolor{white}{+ 0y_1} = 10.\\ 
\end{aligned}
\right.
\]

Векторна форма

\begin{equation*}
\overline{x'} = \left(x_1, x_2, x_3, x_4, x_5, y_1\right)^T
\end{equation*}

\begin{equation*}
\overline{c'} = \left(c_1, c_2, c_3, c_4, c_5, c_6\right) = (-2,-1,0,0,0, M)
\end{equation*}

\begin{equation*}
L' = \overline{c'}\overline{x'} \rightarrow \min
\end{equation*}

\begin{equation*}
A' = \Vert a_{ij} \Vert = 
\begin{pmatrix}
	1 & 2 & 1 & 0 & 0 & 0\\
	5 & 7 & 0 & -1 & 0 & 1\\
	3 & -2 & 0 & 0 & 1 & 0
\end{pmatrix}
\end{equation*}

\begin{equation*}
 \overline{b} = \left( 7, 4, 10 \right)^T
\end{equation*}

\begin{equation*}
 A'\overline{x'} = \overline{b}
\end{equation*}


\section{Модифікований симплекс-метод}

\subsection{Розв'язок М-задачі модифікованим симплекс-методом}

\subsubsection{Допоміжна таблиця}

\begin{table}[H]
    \centering
    \begin{tabular}{|l|l|l|l|l|l|l|l|l|l|l|l|}
    \hline
         & $c_1$ & $c_2$ & $c_3$ & $c_4$ & $c_5$ & $c_6$ & & & & & \\ \hline
         & -2 & -1 & 0 & 0 & 0 & M & & & & & \\ \hline
         $b$ & $A_1$ & $A_2$ & $A_3$ & $A_4$ & $A_5$ & $A_6$ & $U^{0}$ & $U^{1}$ & $U^{2}$ & $U^{3}$ & $U^{4}$ \\ \hline
        7 & 1 & 2 & 1 & 0 & 0 & 0 & 0 & 0 & 0 & 0 & -7/8 \\ \hline
        4 & 5 & 7 & 0 & -1 & 0 & 1 & M & -1/7 & -2/5 & 0 & 0 \\ \hline
        10 & 3 & -2 & 0 & 0 & 1 & 0 & 0 & 0 & 0 & -2/3 & -3/8 \\ \hline
        $\Delta_{j}^{0}$ & -2-5M & -1-7M & 0 & M & 0 & 0 & & & & & \\ \hline
        $\Delta_{j}^{1}$ & -9/7 & 0 & 0 & -1/7 & 0 & -1/7 & & & & & \\ \hline
        $\Delta_{j}^{2}$ & 0 & 9/5 & 0 & -3 & 0 & 3 & & & & & \\ \hline
        $\Delta_{j}^{3}$ & 0 & -7/3 & 0 & 0 & 2/3 & 0 & & & & & \\ \hline
        $\Delta_{j}^{4}$ & 0 & 0 & 7/8 & 0 & 3/8 & M & & & & & \\ \hline
    \end{tabular}
\end{table}


\subsubsection{Основна таблиця. Крок №1}

\begin{table}[H]
    \centering
    \begin{tabular}{|l|l|l|l|l|l|l|l|l|}
    \hline
         $c_\text{б}$ & $X_\text{б}$ & $b$ & $B_{1}^{-1}(0)$ & $B_{2}^{-1}(0)$ & $B_{3}^{-1}(0)$ & $A_{2}$ & $\alpha_{2}^{0}$  & $\Theta^{0}$ \\ \hline
        0 & $x_3$ & 7 & 1 & 0 & 0 & 2 & 2 & 7/2 \\ \hline
        M & $x_6$ & 4 & 0 & 1 & 0 & 7 & 7 & 4/7 \\ \hline
        0 & $x_5$ & 10 & 0 & 0 & 1 & -2 & -2 &  \\ \hline
        & & $U^{0}$ & 0 & M & 0 & & &  \\ \hline
    \end{tabular}
\end{table}

\begin{equation*}
	U^{0} = c_{\text{б}}^TB^{-1}(0)
\end{equation*}

\begin{equation*}
	\Delta_{j}^{0} = c_j - U^{0}A_{j}
\end{equation*}

\begin{equation*}
	\Delta_{k}^{0} = \min_{j=1 \dots 6} \Delta_{j}^{0} = \Delta_{2}^{0} = -1-7M
\end{equation*}

\begin{equation*}
	\alpha_{2}^{0} = B^{-1}(0)A_{k}
\end{equation*}

\begin{equation*}
	\Theta_t^{0} = \min_{i: \Theta_i^{0} \geqslant 0} \Theta_i^{0} = \Theta_2^{0} = \frac{4}{7}
\end{equation*}

\begin{equation*}
	\Theta_t^{0} = \frac{b_t}{c_\text{bt}} = \frac{b_t}{c_l} = \frac{b_2}{c_\text{b2}} = \frac{b_2}{c_\text{6}}
\end{equation*}

$t = 2$ - ведучий рядок

$k = 2$ - ведучий стовпець

$a_{tk} = a_{22}$ - ведучий елемент

$l = 6$

$l$-у змінну виводимо з базису і вводимо $k$-у.

\subsubsection{Основна таблиця. Крок №2}

\begin{table}[H]
    \centering
    \begin{tabular}{|l|l|l|l|l|l|l|l|l|}
    \hline
         $c_\text{б}$ & $X_\text{б}$ & $b$ & $B_{1}^{-1}(1)$ & $B_{2}^{-1}(1)$ & $B_{3}^{-1}(1)$ & $A_{1}$ & $\alpha_{1}^{1}$  & $\Theta^{1}$ \\ \hline
        0 & $x_3$ & 41/7 & 1 & -2/7 & 0 & 1 & -3/7 &  \\ \hline
        -1 & $x_2$ & 4/7 & 0 & 1/7 & 0 & 5 & 5/7 & 4/5 \\ \hline
        0 & $x_5$ & 78/7 & 0 & 2/7 & 1 & 3 & 31/7 & 78/31 \\ \hline
        & & $U^{1}$ & 0 & -1/7 & 0 & & &  \\ \hline
    \end{tabular}
\end{table}

\begin{equation*}
	U^{1} = c_{\text{б}}^TB^{-1}(1)
\end{equation*}

\begin{equation*}
	\Delta_{j}^{1} = c_j - U^{1}A_{j}
\end{equation*}

\begin{equation*}
	\Delta_{k}^{1} = \min_{j=1 \dots 6} \Delta_{j}^{1} = \Delta_{1}^{1} = -\frac{9}{7}
\end{equation*}

\begin{equation*}
	\alpha_{1}^{1} = B^{-1}(1)A_{k}
\end{equation*}

\begin{equation*}
	\Theta_t^{1} = \min_{i: \Theta_i^{1} \geqslant 0} \Theta_i^{1} = \Theta_2^{1} = \frac{4}{5}
\end{equation*}

\begin{equation*}
	\Theta_t^{1} = \frac{b_t}{c_\text{bt}} = \frac{b_t}{c_l} = \frac{b_2}{c_\text{b2}} = \frac{b_2}{c_\text{2}}
\end{equation*}

$t = 2$ - ведучий рядок

$k = 1$ - ведучий стовпець

$a_{tk} = a_{21}$ - ведучий елемент

$l = 2$

$l$-у змінну виводимо з базису і вводимо $k$-у.

\subsubsection{Основна таблиця. Крок №3}

\begin{table}[H]
    \centering
    \begin{tabular}{|l|l|l|l|l|l|l|l|l|}
    \hline
         $c_\text{б}$ & $X_\text{б}$ & $b$ & $B_{1}^{-1}(2)$ & $B_{2}^{-1}(2)$ & $B_{3}^{-1}(2)$ & $A_{4}$ & $\alpha_{4}^{2}$  & $\Theta^{2}$ \\ \hline
        0 & $x_3$ & 31/5 & 1 & -1/5 & 0 & 0 & 1/5 & 31  \\ \hline
        -2 & $x_1$ & 4/5 & 0 & 1/5 & 0 & -1 & -1/5 & \\ \hline
        0 & $x_5$ & 38/5 & 0 & -3/5 & 1 & 0 & 3/5 & 38/3 \\ \hline
        & & $U^{2}$ & 0 & -2/5 & 0 & & &  \\ \hline
    \end{tabular}
\end{table}

\begin{equation*}
	U^{2} = c_{\text{б}}^TB^{-1}(2)
\end{equation*}

\begin{equation*}
	\Delta_{j}^{2} = c_j - U^{2}A_{j}
\end{equation*}

\begin{equation*}
	\Delta_{k}^{2} = \min_{j=1 \dots 6} \Delta_{j}^{2} = \Delta_{4}^{2} = -3
\end{equation*}

\begin{equation*}
	\alpha_{4}^{2} = B^{-1}(2)A_{4}
\end{equation*}

\begin{equation*}
	\Theta_t^{2} = \min_{i: \Theta_i^{2} \geqslant 0} \Theta_i^{2} = \Theta_3^{2} = \frac{38}{3}
\end{equation*}

\begin{equation*}
	\Theta_t^{2} = \frac{b_t}{c_\text{bt}} = \frac{b_t}{c_l} = \frac{b_3}{c_\text{b3}} = \frac{b_3}{c_\text{5}}
\end{equation*}

$t = 3$ - ведучий рядок

$k = 4$ - ведучий стовпець

$a_{tk} = a_{34}$ - ведучий елемент

$l = 5$

$l$-у змінну виводимо з базису і вводимо $k$-у.

\subsubsection{Основна таблиця. Крок №4}

\begin{table}[H]
    \centering
    \begin{tabular}{|l|l|l|l|l|l|l|l|l|}
    \hline
         $c_\text{б}$ & $X_\text{б}$ & $b$ & $B_{1}^{-1}(3)$ & $B_{2}^{-1}(3)$ & $B_{3}^{-1}(3)$ & $A_{2}$ & $\alpha_{2}^{3}$  & $\Theta^{3}$ \\ \hline
        0 & $x_3$ & 11/3 & 1 & 0 & -1/3 & 2 & 8/3 & 11/8  \\ \hline
        -2 & $x_1$ & 10/3 & 0 & 0 & 1/3 & 7 & -2/3 & \\ \hline
        0 & $x_4$ & 38/3 & 0 & -1 & 5/3 & -2 & -31/3 & \\ \hline
        & & $U^{3}$ & 0 & 0 & -2/3 & & &  \\ \hline
    \end{tabular}
\end{table}

\begin{equation*}
	U^{3} = c_{\text{б}}^TB^{-1}(3)
\end{equation*}

\begin{equation*}
	\Delta_{j}^{3} = c_j - U^{3}A_{j}
\end{equation*}

\begin{equation*}
	\Delta_k^{3} = \min_{j=1 \dots 6} \Delta_j^{3} = \Delta_2^{3} = -\frac{7}{3}
\end{equation*}

\begin{equation*}
	\alpha_{2}^{3} = B^{-1}(3)A_{2}
\end{equation*}

\begin{equation*}
	\Theta_t^{3} = \min_{i: \Theta_i^{3} \geqslant 0} \Theta_i^{3} = \Theta_3^{3} = \frac{11}{8}
\end{equation*}

\begin{equation*}
	\Theta_t^{3} = \frac{b_t}{c_\text{bt}} = \frac{b_t}{c_l} = \frac{b_1}{c_\text{b1}} = \frac{b_1}{c_\text{3}}
\end{equation*}

$t = 1$ - ведучий рядок

$k = 2$ - ведучий стовпець

$a_{tk} = a_{12}$ - ведучий елемент

$l = 3$

$l$-у змінну виводимо з базису і вводимо $k$-у.

\subsubsection{Основна таблиця. Крок №5}

\begin{table}[H]
    \centering
    \begin{tabular}{|l|l|l|l|l|l|}
    \hline
         $c_\text{б}$ & $X_\text{б}$ & $b$ & $B_{1}^{-1}(4)$ & $B_{2}^{-1}(4)$ & $B_{3}^{-1}(4)$ \\ \hline
        -1 & $x_2$ & 11/8 & 3/8 & 0 & -1/8 \\ \hline
        -2 & $x_1$ & 17/4 & 1/4 & 0 & 1/4 \\ \hline
        0 & $x_4$ & 215/8 & 31/8 & -1 & 3/8 \\ \hline
        & & $U^{4}$ & -7/8 & 0 & -3/8 \\ \hline
    \end{tabular}
\end{table}

\begin{equation*}
	U^{4} = c_{\text{б}}^TB^{-1}(4)
\end{equation*}

\begin{equation*}
	\Delta_{j}^{4} = c_j - U^{4}A_{j}
\end{equation*}

\begin{equation*}
	\Delta_k^{4} = \min_{j=1 \dots 6} \Delta_j^{4} = \Delta_2^{4} = -\frac{7}{3}
\end{equation*}

Оскільки $\min_{j=1 \dots 6} \Delta_j^{4} \geqslant 0$, ми можемо завершити симплекс-метод


\subsubsection{Відповідь}


\begin{equation*}
\text{При }\overline{x'} = \left(x_1, x_2, x_3, x_4, x_5, y_1\right) = \left(\frac{17}{4},\frac{11}{8}, 0, \frac{215}{8}, 0, 0\right)
\end{equation*}

\begin{equation*}
\text{функція }L' = -2x_1 - x_2 + 0x_3 + 0x_4 + 0x_5 + M(y_1) \rightarrow \min,
\end{equation*}

\begin{equation*}
\text{отже }L = -2x_1 - x_2 \rightarrow \min \text{, при } x_1 = \frac{17}{4}, x_2 = \frac{11}{8}
\end{equation*}


\end{document}